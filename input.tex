\documentclass[a4paper, 11pt]{article}
\usepackage[UTF8]{ctex}
\usepackage{CJKutf8}
\usepackage{titlesec}   %设置页眉页脚的宏包
\usepackage{geometry}   %设置页边距的宏包
\usepackage{lastpage}
\usepackage{indentfirst}
\usepackage{graphicx}
\usepackage{float}
\usepackage{booktabs}
\usepackage{threeparttable}
%\setlength{\parindent}{2em}
%设置 上、左、下、右 页边距
\geometry{left=2.5cm,right=2.5cm,top=2.5cm,bottom=2.5cm}
\usepackage{chngcntr}
\counterwithin{table}{section}
\begin{document}
\thispagestyle{empty}
\fbox{
    \parbox[l][700pt][t]{\textwidth}{
        \begin{flushright}
            报告编码:RD2018-SCJN-10
        \end{flushright}
        \parbox[c][170pt][t]{30pt}{}
        \begin{center}
            \huge{
                [(${customer_Information.power_plant_full_name})]\\
                [(${customer_Information.unit_nums})]机组能效报告
                %
            }

            \parbox[c][300pt][t]{30pt}{}

            润电能源科学技术有限公司\\
            二〇一八年十月
        \end{center}
    }
}
\setcounter{page}{0}
\newpagestyle{main}{
\sethead{}{}[(${customer_Information.power_plant_full_name})][(${customer_Information.power_plant_abbreviated_name})]    %设置页眉
    \setfoot{润电能源科学技术有限公司}
    {}
    {第 \thepage 页 共 \pageref{LastPage} 页}      %设置页脚,可以在页脚添加 \thepage  显示页数
    % 添加页眉的下划线
    \headrule
    %添加页脚的下划线
    \footrule
}
\pagestyle{main}    %使用该style
\LARGE{声明:}
\begin{enumerate}
    \item  未经本单位同意不得部分复制。
    \item  仅对样品负责。
    \item  不盖章无效。
\end{enumerate}
\parbox[c][200pt][t]{30pt}{}
\begin{flushright}
    \parbox[c][100pt][l]{300pt}{
        润电能源科学技术有限公司
        \begin{description}
            \item[地址:]河南自贸试验区郑州片区(郑东)\\
						        正光北街40号
            \item[电话:](0371)88923000
            \item[微波:]932126000
            \item[传真:](0371)88923150
            \item[邮编:]450052
            \item[Email:]crp\_nykx\_gw@crpower.com.cn
        \end{description}
    }
\end{flushright}
\newpage
\begin{description}
    \item[项目名称:] \parbox[t]{40em}{华润电力焦作有限公司\\\#1机组能效报告}
    \item[工作时间:] 2018年9月30日~2018年11月15日
    \item[项目负责:] 贾伟
    \item[润电能源科学技术有限公司:]张平安\ 李海荣\ 冯斌
    \item[华润电力焦作有限公司:]\parbox[t]{15em}{张永红\ 刘林\ 李静伟\ 石小兴\\ 关红只\ 刘剑\ 杨翮\ 柴新伟\ 介为民\ 贾秋菊\ 张小鹏}
\end{description}
\parbox[c][100pt][t]{30pt}{}
\huge{
    \begin{description}
        \item[批准:]
        \item[审核:]
        \item[编写:]
    \end{description}
    \begin{center}
        (章)\\
    \end{center}
}
\newpage
\tableofcontents
\newpage
\chapter{
\begin{center}
摘要
\end{center}
}
\addcontentsline{toc}{section}{摘要}
\indent
根据国家节能减排相关政策要求,在华润电力控股对煤电业务技术支持的整体安排下,润电能源科学技术有限公司(以下简称“润电科学”)策划开展煤电机组能效评价工作。通过这项工作的开展,评价机组能效现状,提出改善建议,促进能效提升。\\
\indent
根据中电联《全国火电燃煤机组能效水平对标管理办法》(以下简称“中电联管理办法”)等相关要求,润电科学对[(${customer_Information.power_plant_full_name})](以下简称[(${customer_Information.power_plant_abbreviated_name})])[(${customer_Information.unit_nums})]的供电煤耗、厂用电率、油耗、水耗四个一级能效指标进行评价,同时对影响煤耗的二级指标进行分析。\\
\indent
评价结果如下:
\begin{enumerate}
    \item 供电煤耗:1.	供电煤耗:根据试验结果,[(${customer_Information.unit_nums})]机组额定负荷供电煤耗为[(${rated_coal_consumption_test_value
})] g/kWh;根据{{年度监督月报数据}}监督月报数据,[(${customer_Information.unit_nums})]平均负荷率为[(${coal_consumption.annual_average_load_rate
})],平均供电煤耗为[(${coal_consumption.annual_average_coal_consumption
})]g/kWh,供电煤耗对标如下图。
          \begin{figure}[H]
              \centering
                %\vspace{-1cm} %调整图片与上文的垂直距离
                %\setlength{\abovecaptionskip}{-0.7cm} %调整图片标题与图距离
                %\setlength{\belowcaptionskip}{-0.3cm} %调整图片标题与下文距离
              \includegraphics [width=0.7\textwidth,height=0.35\textheight,trim=0 50 0 50,clip] {fig1.png}
              \caption{供电煤耗(g/kWh)}
              \label{}
          \end{figure}
    \item 厂用电率:根据试验结果,[(${customer_Information.unit_nums})]机组额定工况厂用电率为{{额定试验厂用电率}};根据{{年度监督月报数据}}监督月报统计数据,[(${customer_Information.unit_nums})]机组年统计厂用电率为5.33\%。
          \begin{figure}[H]
              \centering
            \vspace{-1cm} %调整图片与上文的垂直距离
            \setlength{\abovecaptionskip}{-0.7cm} %调整图片标题与图距离
            \setlength{\belowcaptionskip}{-0.3cm} %调整图片标题与下文距离
              \includegraphics [width=0.7\textwidth,height=0.35\textheight,trim=0 50 0 50,clip] {fig2.png}
              \caption{厂用电率}
              \label{}
          \end{figure}
    \item 油耗:根据2017.9~2018.8 监督月报数据,\#1 机组油耗371.8t/a,比行业平均值136.67t/a 高235.13t/a。
    \item 水耗:根据2017.9~2018.8 监督月报数据,\#1 机组平均发电综合耗水率为0.24kg/kWh,比同类型机组行业平均值(0.23kg/kWh)偏高0.01kg/kWh,比GB/T18916.1-2012《取水定额第1部分:火力发电》规定的取水定额(0.46kg/kWh)低0.22kg/kWh,比GB/T26925-2011《节水型企业火力发电行业》规定的节水型企业技术考核指标(≤0.33kg/kWh)低0.09kg/kWh。
	\item 试验供电煤耗与统计煤耗适配性: 根据试验报告供电煤耗与负荷特性曲线,负荷率在[(${coal_consumption.annual_average_load_rate
})]下的供电煤耗为{{试验特性曲线煤耗}}。考虑到[(${customer_Information.unit_nums})]供热、真空、低省、吹灰、启停等影响因素,修正后的试验煤耗为{{试验特性曲线煤耗修正}} g/kWh,比统计供电煤耗[(${coal_consumption.annual_average_coal_consumption
})] g/kWh低{{煤耗适配差值}} g/kWh,两者适配性良好。
    \item 汽轮机热耗率: [(${customer_Information.unit_nums})]机组在额定负荷下修正后的热耗率为{{热耗率}} kJ/kWh,热耗率对标如下图。
    \begin{figure}[H]
    \centering
        \vspace{-1cm} %调整图片与上文的垂直距离
        \setlength{\abovecaptionskip}{-0.7cm} %调整图片标题与图距离
        \setlength{\belowcaptionskip}{-0.3cm} %调整图片标题与下文距离
        \includegraphics [width=0.7\textwidth,height=0.35\textheight,trim=0 50 0 50,clip] {fig3.png}
        \caption{机组热耗率(kJ/kWh)}
        \label{}
    \end{figure}
    \item 锅炉效率:[(${customer_Information.unit_nums})]锅炉额定负荷下效率为{{锅炉效率}}\%,锅炉效率对标如下图。
    \item 排烟温度:\#1 锅炉额定负荷下排烟温度(修正后)为128.4℃,比设计值高7.40℃。
    \item 灰渣含碳量:[(${customer_Information.unit_nums})]锅炉额定负荷下飞灰和炉渣含碳量分别为{{飞灰含碳量}}\%和{{炉渣含碳量}}\%,影响煤耗降低3.66g/kWh。
	\item 排烟中的CO含量:[(${customer_Information.unit_nums})]排烟中的CO含量为{{烟气CO}}16.1 ppm,影响煤耗升高0.02g/kWh。
    \item 空预器性能:\#1机组额定负荷下A、B侧空预器漏风率分别为{{A侧漏风率}}4.44\%和{{B侧漏风率}}4.61\%;烟气侧差压分别为{{A侧差压}}1100 Pa和{{B侧差压}}1050 Pa。
    \item 主汽温度:主汽温度为{{主汽温度}}℃,影响煤耗降低0.03g/kWh。
    \item 再热温度:再热温度为{{再热温度}}℃,影响煤耗升高0.08g/kWh。
    \item 高压缸:高压缸效率为{{高缸效率}}\%,影响煤耗升高0.62g/kWh。
    \item 中压缸:中压缸效率为{{中缸效率}}\%,影响煤耗升高1.03g/kWh。
    \item 加热器端差:\#1、\#2和\#3高压加热器上端差分别为{{#1高加上端差}}1.6℃、{{#2高加上端差}}5.5℃和{{#3高加上端差}}4.1℃,合计影响煤耗升高0.09g/kWh。
\end{enumerate}
\newpage
\section{概述}
\indent
[(${customer_Information.power_plant_abbreviated_name})] [(${customer_Information.unit_nums})]机组投产于2014年12月17日。汽轮机为东方电气集团东方汽轮机有限公司制造的660MW等级、超超临界、一次中间再热、四缸四排汽、双背压、抽汽凝汽式、双抽可调整机组。高、中压缸采用分缸结构,低压缸为对称分流式,机组型号为CC660/605-25/1.0/0.375/600/600。机组热力系统采用单元制方式,共设有八段抽汽分别供给三台高压加热器、一台除氧器和四台低压加热器。给水泵驱动方式:2×50\%B-MCR汽动给水泵,小汽机用汽由四抽供给,引风机驱动方式为汽动,小汽机用汽由四抽供给。\\
\indent
锅炉为哈尔滨锅炉厂生产的超超临界变压运行直流炉,单炉膛、一次再热、平衡通风、固态排渣、全钢构架、前后墙对冲燃烧、不带启动循环泵、露天布置、全悬吊钢结构π型炉,锅炉型号为HG-2039/26.15-PM4,锅炉设计煤种为山西长治煤,以潞安烟煤作为校核煤种;采用微油点火系统,微油油枪采用空气雾化技术。前后墙各布置4层共32只低NOx轴向旋流煤粉燃烧器,对应4套双进双出钢球磨直吹式制粉系统,烟风系统包括两台50\%容量的动叶调节轴流式送风机,两台50\%容量的动叶调节轴流式一次风机,两台50\%容量的静叶调节轴流式汽动引风机,一台30\%容量的静叶可调轴流式电动引风机以及两台三分仓回转式空气预热器。\\
\indent
\#1 机组采用带冷却塔的闭式循环冷却方式,发电用水主要为博爱污水处理厂中水,厂区生活水为引丹渠水。
烟气脱硝装置采用选择性催化还原脱硝(SCR)法,双烟道双反应器,反应器布置在省煤器和空预器之间,烟气脱硝装置采用板式催化剂,按“2+1”层布置催化层,还原剂采用液氨脱硝。通过炉内低氮燃烧技术,控制炉膛出口NOx浓度。在设计煤种、BMCR工况下、处理100\%烟气量的情况下,脱硝效率不低于91\%,脱硝装置出口的烟气中NOX含量不高于45mg/m3。\\
\indent
脱硫装置由山东三融环保工程有限公司总承包建设,采用石灰石-石膏湿法烟气脱硫工艺,采用一炉一塔配置,无GGH,无烟气旁路,脱硫剂采用湿磨制浆系统。设计脱硫效率不低于98.8\%。设计煤种含硫量1.6%,校核煤种含硫量2.5%,吸收塔配五台浆液循环泵、三台氧化风机。超低排放改造时,煤种按照收到基含硫量1.6\%设计,吸收塔内增加了旋汇耦合器、管束式除尘装置。\\
\indent
电除尘器采用福建龙静环保有限公司生产的干式、卧式、板式双室四电场。采用“零风速断电振打”技术,可有效降低除尘器内粉尘的二次扬尘,一、二电场为高频电源,三、四电场为工频加杭州天明环保提供的脉冲电源供电。
\#1机组于2015年9月进行C级检修,汽机侧主要进行以下工作:①高压后汽封进汽管道法兰换垫、紧固;② A、B给水泵小机揭缸大修;③\#1、\#3高加部分系统解体检修;④部分阀门解体检修等。2016年10月进行了C级检修,汽机侧主要进行以下工作:①\#2高加揭盖检查;② 冷却塔填料检查处理。2018年3月进行了C级检修,汽机侧主要进行以下工作:①高低旁减温水阀门解体检修;②高加揭盖检查隔板,管束焊口,灌水消漏。\\
\indent
为了掌握\#1机组能效水平,进一步提高机组运行经济性,润电科学组织汽机、锅炉、环化等专业人员,共同对\#1机组进行了能效评价。
\section{编制依据}
\begin{enumerate}
\item 中电联《全国火电燃煤机组能效水平对标管理办法》(2016版)
\item 《节水型企业火力发电行业》(GB/T26925-2011)
\item 《煤电节能减排升级与改造行动计划》(2014-2020)
\item 水和蒸汽的性质国际协会《水和蒸汽的性质》(IAPWS-IF97)
\item 电站锅炉性能试验规程(GB/T 10184-2015)
\item 《汽轮机热力性能验收试验规程 》(GB/T8117.2-2008)
\item 华润电力焦作有限公司机组集控运行规程、除灰脱硫规程(B版-2016)
\item 华润电力焦作有限公司技术监督月报(2017.10-2018.9)
\item 华润电力焦作有限公司\#1机组诊断性热力性能及煤耗试验报告(2018.08)。
\item 华润电力焦作有限公司\#1汽轮机热力性能考核试验报告(2015.04)。
\item 华润电力焦作有限公司\#1机组节能一体化评估锅炉送风机改造前性能摸底试验报告(2018.5)
\item 华润电力焦作有限公司\#1机组节能一体化评估锅炉热效率、空预器漏风、氧量比对标定试验报告(2018.8)
\item 2016年度全国火电600 MW级机组能效对标及竞赛资料(辅机耗电率)
\item 2017年度全国火电机组能效水平对标资料
\item 《华润电力焦作有限公司全厂水务管理试验研究报告》(2018.1)
\item 《取水定额 第1部分:火力发电行业》(GB/T 18916.1-2012)
\item 机组设备设计资料
\end{enumerate}
\section{评价内容与项目}
\begin{table}[H]
\centering
\caption{能效指标表}  %表格标题
\begin {tabular}{|c|p{30em}|}
\hline
一级指标&二级指标\\
\hline
供电煤耗&热耗率、锅炉效率、主汽压力、主汽温度、再热温度、凝汽器端差、真空严密性、轴封漏汽、加热器端差、减温水、排烟温度、空预器漏风、灰渣含碳量、负荷率、高压缸效率、中压缸效率、低压缸效率\\
水耗&好水率\\
\hline
厂用电率&辅机耗电率\\
\hline
油耗&助燃、点火用油\\
\hline
\end{tabular}
\end{table}
\newpage
\section{评价数据分析}
\subsection{试验数据与分析}
\subsubsection{汽机专业}
润电科学于2018年8月进行了[(${customer_Information.power_plant_abbreviated_name})] [(${customer_Information.unit_nums})]汽轮机性能试验,主要试验结果和数据如表\ref{汽轮机性能试验结果}所示。
\begin{table}[H]
\centering
\caption{\#1汽轮机性能试验结果}  %表格标题
\label{汽轮机性能试验结果}
\begin {tabular}{|c|c|c|c|c|c|c|c|}
\hline
项目&单位&设计值&\multicolumn{ 5 }{c|}{试验值}\\
\hline
发电机功率&kW&1008000  &801838  &799070  &901439  &999727  &980987 \\
\hline
主汽压力&Mpa&25  &19.838  &20.184  &222.6185  &24.844  &25.132 \\
\hline
主汽温度&℃&600  &601.1  &601.4  &600.8  &602  &601.8 \\
\hline
高排压力&Mpa&4.74  &3.837  &3.816  &4.326  &4.814  &4.704 \\
\hline
高排温度&℃&344.5  &356.9  &353.75  &352.9  &353.5  &348.8 \\
\hline
再热压力&Mpa&4.261  &3.525  &3.503  &3.976  &4.425  &4.325 \\
\hline
再热温度&℃&600  &603.3  &603.4  &602.7  &603  &603.1 \\
\hline
中排压力&Mpa&1.111  &0.903  &0.898  &1.009  &1.112  &1.086 \\
\hline
中排温度&℃&392.7  &395.2  &395.6  &393.1  &391.4  &391.7 \\
\hline
排气压力&Mpa&4.9  &5.633  &5.657  &6.346  &6.709  &6.777 \\
\hline
给水温度&℃&296.7  &285  &284.3  &292  &298.8  &296.9 \\
\hline
高压缸效率&\%&88.46  &85.47  &85.6  &85.46  &85.59  &86.63 \\
\hline
中压缸效率&\%&92.38  &91.19  &91.29  &91.34  &91.46  &91.39 \\
\hline
试验热耗率&kJ/kWh&7324  &7651.02  &7628.94  &7625.76  &7621.73  &7611.73 \\
\hline
修正热耗率&kJ/kWh&  &  &  &  &  & \\
\hline
修正发电机功率&kW&1008000  &798743  &796175  &901450  &1007409  &978408 \\
\hline
\end{tabular}
\end{table}
\subsubsection{汽机专业}
润电科学于2018年5月至8月进行了[(${customer_Information.unit_nums})]锅炉性能试验,试验负荷分别为630 MW、500 MW和330 MW,试验结果如表\ref{锅炉性能试验结果}所示;
\begin{table}[H]
\centering
\caption{\#1锅炉性能试验结果}  %表格标题
\label{锅炉性能试验结果}
\begin {tabular}{|p{10em}|c|c|c|c|c|c|c|}
\hline
项目&单位&630 MW工况&500 MW工况&330 MW工况\\
\hline
排烟温度(修正前)	&℃	&139.81 	&140.17 	&129.04 \\
\hline
排烟温度(修正后)	&℃	&118.76	&114.57	&107.51\\
\hline
飞灰含碳量	&\%	&2.27	&2.08	&3.58\\
\hline
炉渣含碳量	&\%	0.24	&2.21	&2.16\\
\hline
空预器出口CO含量	&ppm	&16.1	&2.8	&16.6\\
\hline
排烟热损失q2	&\%	&4.261 	&4.294&4.752 \\
\hline
气体未完全燃烧热损失q3	&\%	&0.006 	&0.001 	&0.008 \\
\hline
固体未完全燃烧热损失q4	&\%	&0.480 	&1.214 	&2.014 \\
\hline
锅炉散热损失q5	&\%	&0.368 	&0.431 	&0.682 \\
\hline
灰、渣物理显热损失q6	&\%	&0.053 	&0.127 	&0.119\\
\hline
其他热损失qoth	&\%	&0.000 	&0.000 	&0.000 \\
\hline
外来热量与燃料低位发热量的百分比qex	&\%	&0.059 	&0.117 	&0.137\\
\hline
换算到保证条件下锅炉热效率η	&\%	&94.889 	&94.050 &	92.560 \\
\hline
A侧空预器漏风率	&\%	&4.44	&/	&/\\
\hline
B侧空预器漏风率	&\%	&4.61	&/	&/\\
\hline
A侧空预器烟气侧阻力	&Pa	&1100	&/	&/\\
\hline
B侧空预器烟气侧阻力	&Pa	&1050	&/	&/\\
\hline
\end{tabular}
\end{table}
\subsubsection{环保专业}
[(${customer_Information.power_plant_abbreviated_name})] [(${customer_Information.unit_nums})]机组脱硫、脱硝系统主要试验数据见表\ref{脱硫、脱硝系统主要试验结果}。
\begin{table}[H]
\centering
\caption{\#1脱硫、脱硝系统主要试验结果}  %表格标题
\label{脱硫、脱硝系统主要试验结果}
\begin {tabular}{|p{10em}|c|c|c|c|c|}
\hline
\multicolumn{ 5 }{|c|}{脱硫系统}\\
\hline
参数名称&	单位&	设计值&	\multicolumn{ 2 }{|c|}{试验值}\\
\hline
锅炉负荷&MW&660&630&500\\
\hline
烟气流量(标干、6\%O2)&m3/h&1988820&/&/\\
\hline
原烟气SO2浓度&mg/m3&4427&2682.5&2836.5\\
\hline
总排口SO2浓度&mg/m3&35&16.1&19.0\\
\hline
脱硫效率&\%&99.21&99.4&99.3\\
\hline
循环泵投运数量&台&5&3(BCD)&3(ACD)\\
\hline
脱硫系统阻力&Pa&2350&1850&1487\\
\hline
	\multicolumn{ 5 }{|c|}{脱硝系统}\\
\hline
脱硝前烟气NOX浓度&mg/m3&500&572.39(A侧)&453.17(B侧)\\
\hline
脱硝后烟气NOx浓度&mg/m3&45&20.52(A侧)&65.96(B侧)\\
\hline
脱硝后A侧烟气NOx浓度分布偏差&\%&&38.47&61.73\\
\hline
脱硝效率&\%&91&96.4&85.4\\
\hline
氨逃逸浓度&ppm&3&/&/\\
\hline
\end{tabular}
\end{table}
\subsubsection{数据汇总表}
根据以上试验数据分析各因素对煤耗的影响,如表4-7和图4-4所示。
\begin{table}[H]
\centering
\begin{threeparttable}[b]
\caption{[(${customer_Information.unit_nums})]机组额定工况下煤耗影汇总表}  %表格标题
\label{机组额定工况下煤耗影汇总表}
\begin {tabular}{|p{8em}|c|c|c|c|p{4em}|p{4em}|p{5em}|}
\hline
项目&单位&设计值&试验值&差值&煤耗上升(g/kWh)&煤耗下降(g/kWh)&可控/不可控因素\\
\hline
发电机功率&kW&660062&633860&/&/&/&/\ \\
\hline
主汽温度&℃&600&600.3&+0.3&/&-0.03&可控\\
\hline
再热温度&℃&600.0&598.9&-1.1&+0.08&/&可控\\
\hline
高压缸效率&\%&84.79&83.32&-1.47&+0.62 &/&不可控\\
\hline
中压缸效率&\%&92.03&90.33&-1.70&+1.03 &/&不可控\\
\hline
高压加热器上端差&℃&-1.7/0/0&1.6/5.5/4.1&3.3/5.54.1&+0.09&/&不可控\\
\hline
低压缸效率、热力系统严密性&/&/&/&/&+2.83&/&/\\
\hline
排烟温度&℃&120&118.76&-1.24&/&-1.98&可控\\
\hline
灰/渣含碳量&\%&/&2.27/0.24&/&/&-3.66&可控\\
\hline
排烟中CO含量&ppm&/&16.1&/&+0.02&/&可控\\
\hline
其他影响因素&/&/&/&/&+0.5&/&不可控\\
\hline
合计&g/kWh&/&/&/&5.17&-5.67&/\\
\hline
\end{tabular}
   \begin{tablenotes}
     \item(注:“+” 表示煤耗增加,“-” 表示煤耗降低)
   \end{tablenotes}
  \end{threeparttable}
\end{table}
\subsection{监督月报数据分析}
\subsubsection{汽机专业}
根据[(${customer_Information.power_plant_abbreviated_name})]技术监督月报,  [(${customer_Information.unit_nums})]汽轮机{{年度监督月报数据}}运行指标见表4-9、表4-10。
\begin{table}[H]
\centering
\caption{\#1汽轮机2017.10~2018.9月运行指标(一)}  %表格标题
\label{汽轮机2017.10~2018.9月运行指标1}
\begin {tabular}{|c|c|c|c|c|c|c|c|}
\hline
项目&负荷率&发电煤耗&供电煤耗&厂用电率&真空度&主汽压力&主汽温度\\
\hline
单位&\%&g/kWh&g/kWh&\%&\%&MPa&℃\\
\hline
201710&75.38&284.35&297.14&4.30&93.46&17.53&597.73\\
\hline
201711&62.22&291.50&306.94&5.03&94.48&14.56&597.43\\
\hline
201804&46.87&308.04&333.98&7.77&95.25&11.47&593.59\\
\hline
201805&54.86&292.21&307.34&4.92&94.80&12.54&593.45\\
\hline
201806&66.30&290.04&303.79&4.52&93.28&16.40&594.34\\
\hline
201807&64.31&293.04&309.57&5.34&92.18&17.95&596.05\\
\hline
201808&61.83&287.14&304.43&5.68&92.01&15.98&596.86\\
\hline
201809&58.93&292.13&307.67&5.05&94.00&15.38&599.25\\
\hline
201712&46.58&249.02&265.46&6.20&96.26&10.46&595.17\\
\hline
201801&69.80&265.89&277.44&4.16&96.26&14.19&592.39\\
\hline
201802&61.20&280.08&293.86&4.69&96.06&15.36&593.61\\
\hline
201803&64.88&266.36&278.87&4.49&95.86&15.90&594.51\\
\hline
年平均&60.62&265.34&278.91&4.88&96.11&13.98&593.92\\
\hline
\end{tabular}
\end{table}
\begin{table}[H]
\centering
\caption{\#1汽轮机2017.10-2018.9月运行指标(二)}  %表格标题
\label{汽轮机2017.10-2018.9月运行指标2)}
\begin {tabular}{|c|c|c|c|c|c|c|c|}
\hline
项目&再热压力&再热温度&利用小时数&循环水泵耗电率&凝结水泵耗电率&前置泵耗电率\\
\hline
单位&MPa&℃&h&\%&\%&\%\\
\hline
201710&3.46&599.28&744.00&0.82&0.17&0.17\\
\hline
201711&2.77&596.47&540.93&0.86&0.18&0.21\\
\hline
201804&2.20&591.87&148.30&1.45&0.27&0.28\\
\hline
201805&2.49&596.99&744.00&1.05&0.23&0.22\\
\hline
201806&3.29&596.19&720.00&0.92&0.19&0.19\\
\hline
201807&3.47&594.59&473.00&1.23&0.20&0.21\\
\hline
201808&3.05&594.85&311.57&1.18&0.19&0.21\\
\hline
201809&2.97&592.04&491.28&0.74&0.20&0.22\\
\hline
年平均&2.96&595.29&521.64&1.03&0.20&0.21\\
\hline
201712&1.98&594.02&327.62&0.98&0.21&0.27\\
\hline
201801&2.75&591.28&738.93&0.60&0.17&0.19\\
\hline
201802&2.91&595.71&672.00&0.69&0.17&0.20\\
\hline
201803&3.05&594.48&527.65&0.66&0.17&0.20\\
\hline
年平均&2.67&593.87&566.55&0.73&0.18&0.21\\
\hline
\end{tabular}
\end{table}
\subsubsection{锅炉专业}
根据焦作项目技术监督月报,\#1锅炉{{年度监督月报数据}}月运行指标见表4-11、4-12。
\begin{table}[H]
\centering
\caption{\#1锅炉2017.10~2018.09月运行指标(一)}  %表格标题
\label{锅炉2017.10~2018.09月运行指标(一)}
\begin {tabular}{|c|c|c|c|c|c|c|c|}
\hline
项目&锅炉效率&排烟温度&冷风温度&飞灰含碳量&炉渣含碳量&空预器漏风率&空预器烟气差压\\
\hline
单位&\%&℃&℃&\%&\%&\%&Pa\\
\hline
2017.10&93.85&134.64&15.54&1.44&3.86&3&1344.88\\
\hline
2017.11&93.76&134.49&15.62&1.74&4.23&3&1328.50\\
\hline
2017.12&92.54&127.03&6.50&1.64&0.98&3&964.39\\
\hline
2018.01&94.31&130.38&1.61&3.49&2.74&3.1&1272.72\\
\hline
2018.02&92.16&137.96&5.87&2.16&2.25&3&1537.91\\
\hline
2018.03&92.09&137.20&10.06&4.05&4.20&3&1545.23\\
\hline
2018.04&93.41&114.37&21.15&3.05&4.09&3&473.43\\
\hline
2018.05&93.33&130.92&22.56&2.39&3.54&3.29&556.25\\
\hline
2018.06&92.72&137.82&27.33&3.56&3.94&3.29&714.45\\
\hline
2018.07&92.72&127.54&28.54&4.30&7.57&3.29&727.26\\
\hline
2018.08&92.72&136.13&30.75&2.95&8.67&3.29&652.51\\
\hline
2018.09&92.59&130.71&21.43&2.61&5.42&3.29&665.51\\
\hline
平均&93.02&131.60&17.25&2.78&4.29&3.13&981.92\\
\hline
先进值&/&/&/&/&/&&/\\
\hline
平均值&/&/&/&/&/&&/\\
\hline
\end{tabular}
\end{table}
\begin{table}[H]
\centering
\caption{\#1锅炉2017.10~2018.09月运行指标(二)}  %表格标题
\label{锅炉2017.10~2018.09月运行指标(二)}
\begin {tabular}{|c|c|p{4em}|p{4em}|p{3em}|p{4em}|p{3em}|c|}
\hline
项目&油耗&过热减温水流量&再热减温水流量&磨煤机耗电率&一次风机耗电率&送风机耗电率&引风机耗电率\\
\hline
单位&t&t/h&t/h&\%&\%&\%&\%\\
\hline
2017.10&0&46&12.39&0.96&0.31&0.23&0.00\\
\hline
2017.11&70.56&41&7.47&0.94&0.34&0.23&0.00\\
\hline
2017.12&8.04&44&3.26&1.00&0.42&0.24&0.00\\
\hline
2018.01&1.32&53&2.65&0.88&0.34&0.20&0.00\\
\hline
2018.02&0&88&2.12&1.06&0.38&0.20&0.00\\
\hline
2018.03&2.4&53&0.83&0.96&0.37&0.21&0.00\\
\hline
2018.04&70.8&41&0.39&0.97&0.45&0.23&0.00\\
\hline
2018.05&59.22&71&2.04&0.93&0.38&0.20&0.00\\
\hline
2018.06&9.84&73&1.31&0.91&0.37&0.20&0.00\\
\hline
2018.07&29.96&61&0.98&0.86&0.39&0.20&0.00\\
\hline
2018.08&0.96&76&0.40&0.84&0.39&0.21&0.00\\
\hline
2018.09&99.54&37&1.25&0.94&0.41&0.20&0.00\\
\hline
平均&352.64&57.07&2.92&0.94&0.38&0.21&0.00\\
\hline
先进值&/&/&/&0.33&0.36&0.14&0.48\\
\hline
平均值&69.65&/&/&0.37&0.44&0.17&0.73\\
\hline
\end{tabular}
\end{table}
\subsubsection{化学专业}
[(${customer_Information.unit_nums})]机组采用带冷却塔的闭式循环冷却方式,发电用水主要为博爱污水处理厂中水,厂区生活水为引丹渠水。2017年8月~2018年1月,完成全厂水务管理试验研究,并实施了多项节水改造。
\begin{table}[H]
\centering
\caption{\#1机组综合水耗状况}  %表格标题
\label{机组综合水耗状况}
\begin {tabular}{|c|c|c|c|}
\hline
项目&单位&[(${customer_Information.unit_nums})]机组&备注\\
\hline
单位发电量取水量&kg/kWh&1.43&{{年度监督月报数据}},发电量加权统计 \\
\hline
\end{tabular}
\end{table}
\begin{table}[H]
\centering
\caption{GB/T18916.1、GB/T26925及中电联全国火电机组统计指标}  %表格标题
\label{GB/T18916.1、GB/T26925及中电联全国火电机组统计指标}
\begin {tabular}{|c|c|c|p{15em}|}
\hline
项目&单位&指标&备注\\
\hline
取水定额&kg/kWh&2.40&单机容量600MW级及以上循环冷却机组;依据GB/T 18916.1-2012
《取水定额 第1部分:火力发电》\\
\hline
节水型企业考核指标&kg/kWh&≤1.68&依据GB/T 26925-2011《节水型企业 火力发电行业》。\\
\hline
中电联全国统计平均值&kg/kWh&1.87&\\
\hline
\end{tabular}
\end{table}
\subsubsection{环保专业}
\large{电耗分析}\\
根据{{年度监督月报数据}}监督月报数据统计,脱硫、电除尘系统耗电率如图4-6所示。
\large{物耗分析}
(1)单位脱硫吸收剂消耗
根据现场试验及DCS历史数据计算结果,单位石灰石(或其它)消耗量为{{石灰石单耗}},单位石灰石消耗分级为{{石灰石单耗评级}}级。
根据现场试验及DCS历史数据计算结果,单位氨(尿素)消耗量为{{氨单耗}},单位石灰石消耗分级为{{氨单耗评级}}级。

\section{问题分析与建议}
5.1再热蒸汽温度偏低。根据润电科学2018年8月性能试验数据显示,\#1机组在试验期间再热蒸汽分别为598.9℃,比设计值600℃偏低1.1℃。经计算,影响煤耗0.08 g/kWh,建议:优化燃烧,提高蒸汽参数。\\
\indent
5.2在630 MW负荷试验工况下,机组的高、中压缸效率偏低。高压缸效率为83.32\%,比设计值 84.79\%低 了1.47个百分点。经计算,煤耗升高0.62 g/kWh;中压缸效率为90.33\%,比设计值92.03\%高了1.7个百分点。经计算,煤耗升高1.03g/kWh;缸效率下降共影响煤耗3.27 g/kWh。试验期间发现主机前后汽封出现向外冒汽的现象,前后汽封漏汽量偏大。2018年3月焦作电厂对\#1机组进行了C级检修,检修期间对汽轮机进行揭缸检查,检查发现中压前汽封螺栓断裂,经测量,中压前汽封汽封间隙为2mm,造成高品质蒸汽的浪费,极大影响了机组的安全性和经济性,汽封漏汽是造成缸效率下降的主要原因。建议:调整汽封间隙,减小汽封漏汽。
5.3 高压加热器上端差大。\#1、\#2和\#3高压加热器上端差分别为1.6℃、5.5℃和4.1℃,比设计值高了3.3℃、5.5℃和4.1℃。经计算,煤耗分别升高0.01 g/kWh,0.03 g/kWh和0.05g/kWh,高压加热器端差共影响煤耗0.09 g/kWh。分析可能原因:(1)加热面结垢,增大了传热热阻;(2)加热器汽空间聚集了空气,增大传热热阻;(3)水室隔板泄漏。建议:利用停机检修机会对\#1、\#2、\#3高压加热器进行解体检查。\\
\indent
5.4热力系统严密性较差。试验期间发现多处阀门存在内漏,阀门内漏直接增加了凝汽器的热负荷,造成低压缸效率偏低。见表附表6。建议:①利用停机机会对高低压旁路阀门解体检修,检查密封面是否有杂物和焊渣,重新堆焊、加工后再进行研磨密封面或者更换高质量密封垫;②减少启机旁路投运时间以减少热力系统泄漏;③对其他泄漏阀门逐个排查,找出泄漏原因并处理。经估算,阀门治理后,提高低压缸效率,煤耗降低约2.83 g/kWh。
5.5排汽压力高。在试验期间630 MW工况下,汽轮机排汽压力为10.002 kPa,比设计值4.9 kPa高了5.102 kPa。润电科学对冷端设备进行了排查发现,机组的真空严密性试验合格,凝汽器端差为3.1℃,均在正常运行范围内。实测冷却塔出水温度为33.9℃,环境湿球温度为26.9℃,冷却幅高为7℃,根据冷却水塔的试验标准(DL/T 1027)规定,夏季冷测试却幅高不高于7℃。检查冷却塔发现\#1机组冷却水塔喷淋不均,现场发现有多处较大水柱,进入塔内发现,冷却水塔配水管和喷头堵塞严重,引起进塔循环水无法均匀喷淋,冷却塔内填料出现大面积破损。对比2018年8月月报,\#1机组的循环水入口温度为31.7℃,\#2机组的循环水入口温度为29.24℃,两台机组偏差2.46℃,影响排汽压力为0.976 kPa,影响煤耗2.45 g/kWh。建议:全面检查冷却水塔配水的均匀性以及各母管的泄漏情况;检修中对发现泄漏的母管、墙体进行处理;对损坏的喷嘴、填料进行更换;清除填料中的堵塞物,疏通喷嘴和补充脱落的喷溅盘。\\
\indent
5.6真空泵优化。华润焦作有限公司\#1机组为偏心式双级真空泵,在600MW负荷时运行电流为184A,有较大的节能空间,可改为罗茨真空泵和变频真空泵,经估算,改为罗茨泵年可降低供电煤耗约0.07 g/kWh;改为变频真空泵年可降低供电煤耗约0.03 g/kWh。\\
\indent
5.7引风机背压改造。改造后:将驱动引风机排汽引入到热力循环中,在回收工质的同时,将排汽的热量回收到热力循环的工质中,或将排汽引至热网,将排汽的热量进行回收,减少冷源损失,并且在一定程度上提高了机组的供热方式的灵活性和效率,经估算改造后可降低供电煤耗约1 g/kWh。\\
\indent
	5.8排烟中CO含量偏高。额定负荷试验工况下,排烟中的CO含量偏高,引起煤耗上升0.02 g/kWh。建议电厂在高负荷运行时,进行燃烧优化调整,降低排烟中的CO含量。\\
\indent
5.9根据润电科学试验结果,\#1机组送风机选型偏大,引起其运行效率偏低,电耗偏高。建议对送风机进行更换或变频改造,送风机电耗达到行业先进值时,可降低煤耗0.2 g/kWh左右。\\
\indent
5.10 \#1锅炉年油耗352.64吨,其中点火油耗占74.5\%,而中电联统计同类型机组年平均油耗平均值为69.65吨。建议电厂尽量减少机组启停次数;充分利用临机启动技术,目前先进的临机启动技术已经可以实现蒸汽、凝结水和热风的互通互联,充分利用临机启动技术,可以大大加快机组启动速度,减少投油时间,降低燃油消耗量。机组油耗达到同类型机组平均值时,年平均节约燃油282.99吨。\\
\indent
5.10 脱硫系统耗电率偏高。根据月报数据,脱硫耗电率为1.0\%,高于2016年度行业先进值0.77\%。2017.10 ~2018.9月机组负荷率为61.34\%,入炉煤硫份均值为0.94\%(脱硫设计值1.6\%),负荷率低,入炉煤硫份低于脱硫设计值较多,该工况下脱硫设备有较大节能空间。具体节能建议入下:\\
\indent
(1)对次高层浆液循环泵进行变频改造,在锅炉中低负荷,脱硫入口硫份低于设计值工况下,通过浆液循环泵变频控制对浆液循环泵压力进行微调,实现液气比连续变化,避免了经常频繁启停浆液循环泵(原频繁启停浆液循环泵目的控制排放二氧化硫浓度不能太高也不能太低),提高了环保设施的可靠性,也实现了浆液循环泵节能。估算循环泵改变频后循环泵每小时可节电约 210 kWh。\\
\indent
(2)在上述工况下脱硫塔运行1台氧化风机仍有一定余量,长时间存在能耗过高问题。建议新增一台220kW、流量为5000 m3/h的罗茨式风机,对\#1、2吸收塔氧化风系统母管改造连通,实现一台旧氧化风机+新增氧化风机同时向\#1、\#2吸收塔提供氧化风模式运行,实现氧化风系统节能降耗。经估算氧化风系统改造后,\#1、\#2脱硫系统氧化风机每小时可节电约200 kWh。\\
\indent
(3)球磨机处理能力约25t/h(设计为43.5t/h),制浆系统处理能力低。建议检查球磨机刚球装载量,石灰石旋流器是否偏离设计工况,石灰石来料粒径、纯度是否满足要求等原因,对湿式球磨机制浆系统进行优化调整试验。优化调整后湿式球磨机制浆系统节电约115 kWh。\\
\indent
(4)原石灰石浆浆液泵选型大(流量210 m3/h,电机功率45kW),中低负荷时约三分之二石灰石浆液由回流管路返到石灰石浆液箱,造成石灰石浆液泵能耗浪费,对石灰石浆液泵进行变频改造,取消石灰石浆液泵回流管路,变频改造后石灰石浆液泵每小时节电约25 kWh。\\
\indent
5.11脱硝系统出口NOx浓度场不均匀(见上表4-6、4-7),脱硝后A、B侧烟气NOx浓度分布标准偏差分别达到38.47\%、61.7\%,A、B侧反应器均存在喷氨不均情况,喷氨不均造成氨逃逸,最终引起硫酸氢铵大量生成并粘附在空预器换热表面,造成空预器堵塞。建议电厂进行脱硝系统喷氨优化调整试验,防止空预器堵塞。\\
\indent
5.12脱硫剂及脱硝剂耗量评价。\\
\indent
综上,通过优化燃烧调整,提高蒸汽参数的运行优化手段,机组煤耗有0.1g/kWh节能潜力;通过高加解体检修、阀门内漏治理等检修处理措施,机组煤耗有4.57g/kWh节能潜力;通过真空泵改造、引风机小机背压改造和送风机改造等手段,机组煤耗有1.3 g/kWh增量节能潜力。如表5-1所示。\\

\end{document}